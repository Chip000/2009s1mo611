\section{Teoria da Complexidade Parametrizada}
\label{sec:tcpar}

O artigo \cite{DrumFon} introduz um algoritmo para a obten��o de uma 
solu��o exata do \wa{} usando a Teoria da Complexidade Parametrizada \cite{Downey}.
A Teoria da Complexidade Parametrizada foi recentemente proposta e introduz
a no��o de tratabilidade por par�metro-fixo (FPT) na remodelagem de problemas
dif�ceis, removendo a exponenciabilidade da complexidade do problema original.

O problema de colora��o �tima para um grafo da classe $F - ke$\footnote{
  Classe de grafos que pode ser obtida a partir de $F$ removendo no m�ximo 
  $k$ arestas} � FPT se o problema de colora��o para um grafo da classe $F$
pode ser resolvido em tempo polinomial e o problema para encontrar um 
Modulador\footnote{Conjunto de $k$ arestas que, quando adicionados, 
transformam um grafo $G$ em grafo da classe $F$.} para o grafo da classe 
$F - ke$ for FPT.

O algoritmo apresentado utiliza o fato que a o problema da colora��o 
m�nima de v�rtices em grafos da classe cordal\footnote{Grafos que n�o 
  possuem ciclos induzidos com mais de 3 v�rtices.} pode ser resolvido 
em complexidade de tempo linear no n�mero de v�rtices do grafo. Com isso,
o algoritmo possui uma complexidade de tempo de 
$O\left( \frac{4^k}{(k+1)^{\frac{3}{2}}}(m+n)\right)$, que � uma fun��o 
linear no tamanho da entrada ($n$ e $m$), e uma fun��o exponencial do 
par�metro $k$\footnote{Tamanho do Modulador}.
